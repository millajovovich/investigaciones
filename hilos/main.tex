\documentclass[11pt]{letter}
\usepackage[utf8]{inputenc}
\usepackage[spanish]{babel}
\usepackage{graphicx}
\usepackage{geometry} 
\usepackage{afterpage}
\geometry{
	top=2.5cm, 
	bottom=3cm, 
	left=3cm, 
	right=3cm, 
}
\renewcommand{\baselinestretch}{1.3}

\begin{document}
\begin{center}
\includegraphics[width=0.5\textwidth]{logo-udea.png}\\
\end{center}

Sebastian Balbin Rivera\\
\begin{center}
HILOS
\end{center}
Estos son procesos que parece que se ejecutan al mismo tiempo en la CPU, dándole al computador la virtud de ser multitareas, ejecutando grandes procesos al mismo tiempo, que no es realmente al mismo tiempo, aunque lo parezca. Los hilos fueron una medida a implementar al verse la necesidad de satisfacer más operaciones de un computador al mismo tiempo, dentro del procesador hay núcleos, y con estos bastan para realizar las funciones del sistema, estos pueden ejecutar muchas instrucciones dependiendo de su velocidad de ciclos que realice, mientras más ciclos por segundo mejor. \\\\
Hay ciertos núcleos que se usaban o se usan, que en si, solo se pueden encargase de 1 o dos procesos, por ejemplo, tener una APP abierta en el computador, pero esto presenta limitaciones respecto a realizar varias actividades al mismo tiempo, es aquí donde entra el termino HILOS, que sería el encargado de poder ejecutar varias actividades al mismo tiempo, recordemos que los sistemas están diseñados para diferentes propósitos, esto tiene que ver con el tema ya que hay computadores que tienen restricciones a la hora de encargarle a los núcleos tareas o procesos, pero como fue dicho antes, esto depende del sistema que se tenga, los hilos se pueden encargar de varias funciones en un solo proceso, alivianando así la carga y mejorando el uso de esta por ejemplo en el uso de un juego son muchos procesos, pero los diferentes hilos se pueden encargar uno de la interfaz otro de los procesos que conllevan  esa interfaz y lo que ejecuta el usuario, de esa manera se van repartiendo las funcionalidades para optimizar el funcionamiento.\\\\
El real funcionamiento de estos procesos trabajados no es que se realicen al mismo tiempo, es que internamente a una velocidad increíblemente alta se va saltando de hilo en hilo realizando funciones que cada hilo tiene en su pila, junto con datos de CPU, este último es porque para manejar y repartir los procesos entre estos procesadores lógicos se tienen que obtener información de lo que se va a manejar y como; entonces el proceso consiste en hacer una parte del ejecución de un hilo y salta a otro, realiza de nueva una parte y salta a otro y así sucesivamente, algunos computadores no se enfocan tanto en la tecnología threading, sino mas bien en potenciar los núcleos ya que estos pueden hacer mejor el trabajo que los hilos al tener todos los accesos y propiedades para realizar lo propuesto.



\newpage
REFERENCIAS:\\\\




\end{document}