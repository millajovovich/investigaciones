\documentclass{letter}
\usepackage[utf8]{inputenc}
\usepackage[spanish]{babel}
\usepackage{graphicx}
\usepackage{geometry} 
\usepackage{afterpage}
\geometry{
	top=2.5cm, 
	bottom=3cm, 
	left=3cm, 
	right=3cm, 
}

\renewcommand{\baselinestretch}{1.2}

\begin{document}
\begin{center}
\includegraphics[width=0.5\textwidth]{logo-udea.png}\\
\end{center}

Sebastian Balbin Rivera\\\\
\begin{center}
HILOS
\end{center}
Estos son procesos que parece que se ejecutan al mismo tiempo en la CPU, dándole al computador la virtud de ser multitareas, ejecutando grandes procesos al mismo tiempo, que no es realmente al mismo tiempo, aunque lo parezca. Los hilos fueron una medida a implementar al verse la necesidad de satisfacer más operaciones de un computador al mismo tiempo, dentro del procesador hay núcleos, y con estos bastan para realizar las funciones del sistema, estos pueden ejecutar muchas instrucciones dependiendo de su velocidad de ciclos que realice, mientras más ciclos por segundo mejor. 
Hay ciertos núcleos que se usaban o se usan, que en si, solo se pueden encargase de 1 o dos procesos, por ejemplo, tener una APP abierta en el computador, pero esto presenta limitaciones respecto a realizar varias actividades al mismo tiempo, es aquí donde entra el termino HILOS, que sería el encargado de poder ejecutar varias actividades al mismo tiempo, recordemos que los sistemas están diseñados para diferentes propósitos, esto tiene que ver con el tema ya que hay computadores que tienen restricciones a la hora de encargarle a los núcleos tareas o procesos, pero como fue dicho antes, esto depende del sistema que se tenga, los hilos se pueden encargar de varias funciones en un solo proceso, alivianando así la carga y mejorando el uso de esta


\newpage
REFERENCIAS:\\\\




\end{document}